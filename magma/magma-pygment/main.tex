\documentclass{article}

\title{Magma Pygments}
\author{`Immir'}
\date{March 2023}

\usepackage[T1]{fontenc}
\usepackage[margin=5cc]{geometry}
\parindent=0pt
\parskip=1ex

\usepackage{minted,xcolor}
\usemintedstyle{lovelace}
\definecolor{mintbg}{gray}{0.95}
\setminted{bgcolor=mintbg, frame=lines, rulecolor=\color{gray!50}}
\setminted[magma.py:Magma -x]{mathescape=true} % math mode in comments

\begin{document}
\maketitle

% ----------------------------------------------------------------------
\section{Introduction}

This is a demo of a simple Pygments~\cite{pygments} lexer for the
language of the computer algebra system Magma~\cite{magma}.  Together
with the \LaTeX{} \texttt{minted}~\cite{minted} package, this allows
syntax-highlighted Magma source code to typeset in documents.

In the Appendix one can find the Python source for this lexer --
formatted using the standard Pygments lexer for Python for comparison
with the Magma output in the examples following.

At some point a \texttt{git} repo will likely be published for anyone
wishing to contribute. For the moment, anyone who knows \textit{Immir}
knows how to contact him$\ldots$

% ----------------------------------------------------------------------
\section{Some Magma examples}

Some intrinsics from a small Magma package:

\begin{minted}{magma.py:Magma -x}
// file: intrin.m

intrinsic '@' (x, f::RngUPolElt) -> .
  { Allow user to evaluate polynomials using usual notation: $f(x)$. }
  return Evaluate(f, x);
end intrinsic;

map   := func< f, xs | [ f(x) : x in Eltseq(xs) ] >;
fst   := func< xs | xs[1] >;
snd   := func< xs | xs[2] >;

intrinsic CRT (T::SeqEnum[Tup]) -> .
  { Compute CRT from sequence of pairs <residue,modulus>. }
  return CRT(map(fst,T), map(snd,T));
end intrinsic;

intrinsic PohligHellman (g,h,p::RngIntElt,e::RngIntElt) -> RngIntElt
  { Compute discrete log $x$ where $g^x = h$ and $g$ has order $p^e$. }
  return e gt 1 select x0 + p^e0*x1
      where x1 := recurse(g^(p^e0), h/g^x0,   p, e1)
      where x0 := recurse(g^(p^e1), h^(p^e1), p, e0)
      where e1 := e - e0
      where e0 := e div 2
      where recurse := $$
    else Rep({ d : d in [0..p-1] | g^d eq h });
end intrinsic;

intrinsic PohligHellman (g,h,n::RngIntElt) -> RngIntElt
  { Compute discrete log $x$ from $g^x = h$ where $g$ has (hopefully smooth!) order $n$. }
  return CRT([ <PohligHellman(g^(n div p^e),h^(n div p^e),p,e), p^e>
                where p,e := Explode(f) : f in Factorisation(n) ]);
end intrinsic;

intrinsic PohligHellman (g,h) -> RngIntElt
  { Compute discrete log $x$ from $g^x = h$ where $g$ has hopefully smooth order. }
  return PohligHellman(g,h,Order(g));
end intrinsic;
\end{minted}

Some user code:

\begin{minted}{magma.py:Magma -x}
PSL27 := PermutationGroup< 8 | (2,3,5)(6,7,8), (1,2,4)(3,5,6) >;
S := MatrixAlgebra< FiniteField(2), 3 |
       [ 0,1,0,  1,1,1,  0,0,1 ], [ 1,1,1,  0,1,1,  0,1,0 ] >;
M := GModule(PSL27, S);
M: Maximal;

L := Lattice("E", 8);
S := ShortestVectors(L);
#S; // 120
KissingNumber(L); // 240
w := RSpace(RationalField(), 8) ! [ -1/6, 1/6, -1/2, -1/6, 1/6, -1/2, 1/6, -1/2 ];
C, d := ClosestVectors(L, w);
d; // 8/9
{ Norm(v): v in C }; // { 0, 2 }
{ Norm(v - w): v in C }; // { 8/9 }
\end{minted}

Transcript of a session:

\begin{minted}{magma.py:Magma -x}
> P<x> := PolynomialRing(Rationals());
> f := x^3 + x + 1;
> Evaluate(f,1);
3
> f(1); // this doesn't work?! ridiculous!
>> f(1);
    ^
Runtime error in '@': Bad argument types
Argument types given: RngIntElt, RngUPolElt[FldRat]

> Attach("intrin.m");
> f(1);
3
> g := Integers(2^16384)!3;
> h := g^RandomBits(16384);
> time x := PohligHellman(g,h);
Time: 17.130
> g^x eq h;
true
> PohligHellman(g,g^1337) where g is Random(GL(3,127));
1337
\end{minted}


% ----------------------------------------------------------------------
\section{Installation}

There are at least 3 simple ways to use this ``package'' with the
{\LaTeX} \texttt{minted} package. In all cases, you have to run with
the \texttt{-shell-escape} to \LaTeX{}; \textit{e.g.,}
%
\begin{minted}{bash}
bash$ pdflatex -shell-escape file.tex && xpdf file.pdf
\end{minted}


\subsection{Local copy of lexer}

Place the \texttt{magma.py} file in the working directory and use

\begin{minted}[escapeinside=||]{tex}
  \begin{minted}{magma.py:Magma -x}
    // your magma code here
  \end{minted||}
\end{minted}


\subsection{Execute the lexer script}

If you will be using a lot of \texttt{magma} code snippets, it is
unpleasant to have to use the \texttt{magma.py:Magma -x} name over and
over again. An alternative is to give the \texttt{magma.py} script
execute permission, then in the preamble after loading \texttt{minted}
do the following:

\begin{minted}{tex}
  \renewcommand{\MintedPygmentize}{./magma.py}
\end{minted}

Now your code segments can use the simpler \texttt{magma} format:

\begin{minted}[escapeinside=||]{tex}
  \begin{minted}{magma}
    // your magma code here
  \end{minted||}
\end{minted}


\subsection{Install globally}

Details coming later.


% ----------------------------------------------------------------------
\begin{thebibliography}{9}

\bibitem{pygments} \texttt{https://pygments.org}.

\bibitem{magma} Wieb Bosma, John Cannon, and Catherine Playoust, The
  Magma algebra system. I. The user language, J. Symbolic Comput., 24
  (1997), 235–265.

\bibitem{minted} \texttt{https://github.com/gpoore/minted}.

\end{thebibliography}

\appendix

% ----------------------------------------------------------------------
\section{Magma.py}

Source code for the current version of \texttt{magma.py}.

\inputminted{python}{magma.py}


% ----------------------------------------------------------------------
\section{Tests}

% define any macros needed in documentation
\def\SL{\mathop{\mathrm{SL}}}

Here are some tests based on code extracted from various Magma packages.

\begin{minted}{magma.py:Magma -x}
intrinsic Conjugates(G::Grp, H::Grp: Limit := 10000000) -> {}
  {The set of conjugates of H by elements of G}
  require Type(G) eq Type(H) and H subset G:
	  "Argument 2 is not a subgroup of argument 1";
  N := Normalizer(G, H);
  require Index(G, N) le Limit:
    "Number of conjugates of H in G is more than " cat Sprint(Limit);
  if Type(G) eq GrpPC then
    return { PowerGroup(G) | H^t: t in Transversal(G, N) };
  end if;
  return {H^t: t in Transversal(G, N)};
end intrinsic;

/* Given an ideal corresponding to an absolutely irreducible trace tuple t and hence
 * an absolutely irreducible representation $\Delta\colon F_2 \to \SL(3,K)$, find a tuple of
 * nine words $(w_1, \ldots, w_9)$ such that $(\Delta(w_1), \ldots, \Delta(w_9))$ is a basis of $K^{3 \times 3}$.
 *
 * Note that $(\Delta(w_1), \ldots, \Delta(w_9))$ is a basis if and only if the matrix
 * $(\mathop{tr}(\Delta(w_i)\Delta(w_j)))_{i,j}$ has full rank, hence non-zero determinant.
 */
L3FindBasis := function(I)
  if exists(b){ b : b in possibleBases | GramMatrixDeterminant(b, I) ne 0 } then
    return b;
  end if;
  error "not absolutely irreducible";
end /*safe*/ function;

intrinsic IsInTwistedForm( x::GrpLieElt, c::OneCoC ) -> BoolElt
  {Returns true iff $x\in G(K)$ is an element of the twisted group of Lie type $G_c(k)$}
  G := Parent(x);
  require G cmpeq Domain(Group(GammaGroup(c)))
       : "Parent(x) and the cocycle do not match";
  Gamma, m := ActingGroup(GammaGroup(c));
  return forall{ i : i in [1..Ngens(Gamma)] |
                 x eq x @ FieldAutomorphism(G, m(Gamma.i)) @ c(Gamma.i) };
end intrinsic;

intrinsic Bezoutian(H::HypGeomData) -> RngIntElt // e.g. [1,2,3],[5] -> 5
  {The resultant of the defining polynomials of the hypergeometric data}
  f,g:=DefiningPolynomials(H); return Resultant(f,g); end intrinsic;
\end{minted}


Here are some examples for testing edge cases.

\begin{minted}{magma.py:Magma -x}
intrinsic myintrin1(G :: GrpMat[FldFin]) -> {}, [], {RngIntElt}
  { some documentation }
  return {}, [], 0; end intrinsic;

intrinsic myproc(x,y,z::{})
  { documentation }
end intrinsic;

intrinsic myintrin2(x::{}, y::{[]}) -> {}
  { some documentation }
  return {}; end intrinsic;

intrinsic myintrin3(G:: GrpMat : Degree := "All", Set := {}) -> BoolElt, {}
  { { { { some } more } documentation { with { [ crazy ] braces } } } }
  return false; end intrinsic;
\end{minted}

Here are examples of valid Magma code that fail to format correctly;
these examples are currently considered too extreme to worry about.

\begin{minted}{magma.py:Magma -x}
// incorrectly matches braces within /* */ comment
intrinsic junk(a) /* {} */ { docs }
  return a+1; end intrinsic;

// can't have comment between intrinsic name and open paren
intrinsic junk /* */ (a) /* comment
  */ { doc } printf "a = %o\n", a; end intrinsic;
\end{minted}


\end{document}
